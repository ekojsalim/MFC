\documentclass[10pt,a4paper]{report}
\usepackage{graphicx}
\usepackage{ragged2e}
\usepackage{atbegshi,picture}
\AtBeginShipoutNext{\AtBeginShipoutUpperLeft{%
  \put(\dimexpr\paperwidth-1cm\relax,-1.5cm){\makebox[0pt][r]{\framebox{Ilmu Pengetahuan Teknik}}}%
}}
\usepackage{scrextend}
\addtokomafont{labelinglabel}{\sffamily\bfseries}
\begin{document}
\pagenumbering{gobble}
\begin{titlepage}


	\vspace{1cm}	
	\centering
	\includegraphics[width=0.175\textwidth]{gfx/lipi}\par\vspace{1cm}
	{\huge\bfseries Meningkatkan Efisiensi Pengolahan Limbah Air melalui Proses Hibrida dengan Microbial Fuel Cell\par}
	\vspace{1cm}
	\includegraphics[width=0.3\textwidth]{gfx/narada}\par\vspace{1cm}
	\vspace{1cm}
	{\Large Eko J. Salim \& Gerraldo S. Candra\par}
	\vfill

% Bottom of the page
	{\large 2016\par}
\end{titlepage}
\newpage
\begin{labeling}{Kategori:}
\item [Judul:] Meningkatkan Efisiensi Pengolahan Limbah Air melalui Proses Hibrida dengan Microbial Fuel Cell
\item [Bidang:] Ilmu Pengetahuan Teknik
\item [Kategori:] Bioteknologi
\item [Nama:] Eko J. Salim \& Gerraldo S. Candra
\item [Sekolah:] SMP Narada
\end{labeling}
\rule{\textwidth}{0.4pt}
\begin{labeling}{Metodologi penelitian:}
\item [Objek penelitian:] Pembuatan alat 
\item [Penelitian Lanjutan:] Tidak
\item [Metodologi penelitian:] Kuantitatif
\item [Metode penelitian:] Studi Laboratorium
\end{labeling}
\rule{\textwidth}{0.4pt}
\vspace{1cm}
\begin{center}
\textbf{Abstrak}\\

\justify
\noindent Proses pengolahan limbah air sekarang meninggalkan banyak ruang untuk dikembangkan dan ditingkatkan. Limbah dapat menjadi sumber energi yang berharga --- sekitar 9 kali lebih banyak energi terkandung dalam limbah dibandingkan dengan energi yang diperlukan untuk mengolahnya dengan cara modern. Proses-proses yang tersedia sekerang jarang mempertimbangkan dan menggunakan hal ini, proses-proses ini juga sering tidak efisien dan 'kotor'.
Menggabungkan teknologi-teknologi pengolahan limbah menjadi sebuah proses hibrida berpusat pada Microbial Fuel Cell (MFC) dapat memecahkan masalah ini. Keuntungan yang didapat dalam menggunakan proses hibrida ini diantara lain adalah berikut: ekstraksi sumber daya, ekstraksi energi (listrik) yang lebih besar dan efisien dan pembersihan limbah yang lebih bersih.
\end{center}
\end{document}